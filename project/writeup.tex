\documentclass{article}
\title{CS181: Final Project Writeup}
\author{Danny Zhu \& Tianhui Cai}

\begin{document}
\maketitle


In order to map out the frequency with which nutritious and poisonous
plants appeared in the space, we wrote a strategy ({\tt
  player1.ExploreMoveGenerator}) that moved around the origin in a
spiral pattern, systematically covering all of the spaces near the
origin. By running a few thousand trials, we produced a plot of the
frequencies, shown in figure~\ref{fig:plant_freq}. It appears that
poisonous plants occur with a constant frequency of about .15, while
nuritious plants have some distribution peaking at the same value at
the origin and decreasing with distance away from the origin.

We used a neural network to distinguish the plants, implemented using
an adapted version of the code from assignment 2 (included as {\tt
  nn.py}. Based on image data gathered from the same spiral runs
described above, we gathered 82000 images of nutritious plants and
273000 images of poisonous plants. We used 72000 for the training data
and 6000 each for test and validation sets. After a little
experimentation, it looked like having a hidden layer didn't help
performance, so we went with a simple two-layer neural network. After
20 epochs of training, we obtained a network with a test performance
of $68.3\%$.

\end{document}
