\documentclass{article}
\usepackage{amsmath}

\renewcommand{\labelenumi}{(\alph{enumi})}
\renewcommand{\labelenumii}{(\roman{enumii})}
\title{CS181: Clustering and Parameter Estimation}
\author{Danny Zhu \& Tianhui Cai}
\let\b\mathbf

\begin{document}
\maketitle

\section*{Problem 1}
\begin{enumerate}
\item The set of points satisfying the given constraint is an
  $m$-dimensional hypercube of edge $2\epsilon$, so the probability is
  $(2\epsilon)^m$.
\item Then the set of points satisfying the constraint is the
  intersection of a size $2\epsilon$ hypercube centered at $\mathbf x$
  with the unit hypercube; the volume of that set is at most the
  volume of the small hypercube.
\item 
  \begin{align*}
    d(\b x,\b y)^2&=\sum_j(x_j-y_j)^2\\
    d(\b x,\b y)^2&\ge(x_j-y_j)^2\qquad\textrm{for each $j$}\\
    d(\b x,\b y)&\ge|x_j-y_j|\qquad\textrm{for each $j$}\\
    d(\b x,\b y)&\ge\max_j|x_j-y_j|
  \end{align*}
  Now, $d(\b x,\b y)\le\epsilon$ implies $\max_j|x_j-y_j|\le\epsilon$,
  so the set of $\b y$ satisfying the former is a subset of the set
  satisfying the latter; hence the former has a probability of
  occurring that is not greater.
\item We want a probability of at most $\delta$ that no points are
  nearby. That probability is
\item 
\end{enumerate}
\section*{Problem 2}
\begin{enumerate}
\item 
\item 
\item 
\item 
\item 
\end{enumerate}
\section*{Problem 3}
\begin{enumerate}
\item 
\item 
\item 
\item 
\item 
\end{enumerate}
\section*{Problem 4}
\begin{enumerate}
\item 
  \begin{enumerate}
  \item 
  \item 
  \end{enumerate}
\item 
  \begin{enumerate}
  \item 
  \item 
  \end{enumerate}
\end{enumerate}
\end{document}
