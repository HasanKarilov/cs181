\documentclass{article}
\usepackage{qtree}
\begin{document}
\begin{enumerate}
\item \textbf{Decision Trees and ID3}
  \begin{enumerate}
  \item The result of splitting on $A$:
    \begin{center}
      \Tree [.3:4 1:2 2:2 ]
    \end{center}
    The associated entropy:
    \[-\left(\frac17\ln\frac13+\frac27\ln\frac23+\frac27\ln\frac24+\frac27\ln\frac24\right)\approx.669\]

    And for splitting on $B$:
    \begin{center}
      \Tree [.3:4 2:3 1:1 ]
    \end{center}
    \[-\left(\frac27\ln\frac25+\frac37\ln\frac35+\frac17\ln\frac12+\frac17\ln\frac12\right)\approx.679\]

    So splitting on $A$ provides a result with a slightly lower
    entropy, and hence slightly higher information gain.

    Splitting on $A$ might be more useful because it provides a more
    even separation of the data into the true and false branches;
    splitting on $B$ might be more useful because TODO.
  \item 
    In the following tree, going left indicates a True, and going right indicates a False. 

    \Tree [ .A [.B F [.C F T ] ] [ .B T T ] ]
     
    In the first stage, splitting on A generates the lowest remainder.
    Then considering the other three attributes, if we have A as True, 
    then attributes B and C are tied for lowest remainder; if A is False, 
    again B and C are tied for lowest remainder. We split on B in both cases.
    
    If A is True and B is True, there is only one case, in which the classification
    is False. Otherwise, there are two cases that are exactly dependent on C.

    If A is False and B is True, there is one case, in which the classification is
    True. Otherwise, there are two cases that cannot be separated, in which case
    we arbitrarily pick True. 
 
    6 of 7 are correctly identified by the tree. 

  \item Consider the following tree, that also correctly identifies 6 out of 7.
    Again, going left indicates True, and going right indicates False.

    \Tree [ .B [ .C T F ] [ .C F T ] ]

    We see that the ID3 algorithm does not have any preference for shorter or
    simpler trees. 
 
  \end{enumerate}
\item \textbf{ID3 with Pruning}
  \begin{enumerate}
    \setcounter{enumii}2
  \item  
  \item  
    \begin{enumerate}
    \item TODO: how implementation makes use of instance weight in splitting decisions
 
      AdaBoost on ID3 would not work if splitting decisions were made by counting instances
      instead of summing weights, because AdaBoost is based on the idea that you update
      weights based on which data instances need better fitting. 
    \item Weighted entropy of $\{x_1,\ldots,x_n\}$ where $y_1=T$, $y_i=F$ else, $w_1=0.5$,
      and other weights are $0.5/(n-1)$: by definition
      $entropy(D)=-\sum_{y\in Y} \frac{W_y}{W} \log_2 \frac{W_y}{W}$
      and $W_y=\sum_i w_k(i) I(y_i=y)$; we have $W_{T}=0.5$, $W_{F}=0.5(n-1)^{-1}(n-1)=0.5$, $W=1$,
      and so $$entropy(D)=-\left[ \frac{1}{2}\log_2\frac{1}{2} + \frac{1}{2}\log_2\frac{1}{2} \right] = 1$$
    \end{enumerate}
    \begin{enumerate}
    \item 
    \item 
    \item 
    \item 
    \end{enumerate}
  \end{enumerate}
\item 
\end{enumerate}
\end{document}
