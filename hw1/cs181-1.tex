\documentclass{article}
\usepackage{qtree}
\begin{document}
\begin{enumerate}
\item \textbf{Decision Trees and ID3}
  \begin{enumerate}
  \item The result of splitting on $A$:
    \begin{center}
      \Tree [.3:4 1:2 2:2 ]
    \end{center}
    The associated entropy:
    \[-\left(\frac17\ln\frac13+\frac27\ln\frac23+\frac27\ln\frac24+\frac27\ln\frac24\right)\approx.669\]

    And for splitting on $B$:
    \begin{center}
      \Tree [.3:4 2:3 1:1 ]
    \end{center}
    \[-\left(\frac27\ln\frac25+\frac37\ln\frac35+\frac17\ln\frac12+\frac17\ln\frac12\right)\approx.679\]

    So splitting on $A$ provides a result with a slightly lower
    entropy, and hence slightly higher information gain.

    Splitting on $A$ might be more useful because it provides a more
    even separation of the data into the true and false branches;
    splitting on $B$ might be more useful because TODO.
  \item 
  \item 
  \end{enumerate}
\item \textbf{ID3 with Pruning}
  \begin{enumerate}
    \setcounter{enumii}2
  \item 
  \item 
    \begin{enumerate}
    \item 
    \item 
    \end{enumerate}
    \begin{enumerate}
    \item 
    \item 
    \item 
    \item 
    \end{enumerate}
  \end{enumerate}
\item 
\end{enumerate}
\end{document}
